\documentclass{article}
\usepackage[utf8]{inputenc}
%son muy importantes pq nos dicen el tipo de doc que estamos usando, si colocamos tesis cambia la estructura (en la linea 1) y en la 2da linea el utf8 es para poder poner los acentos y las ñ
\usepackage{lipsum} 

\title{PRUEBA}
\author{Sebastian Zaragoza}
\date{April 2022}

\begin{document}
%Todo lo que esta arriba de la linea 8 se llama preambulo

\maketitle

\section{Introduction}
\lipsum[2-4] %El paquete Lipsum tiene acceso a 150 párrafos de texto ficticio de "Lorem ipsum". `lipsum[2-4]` imprimirá lorem ipsum párrafo 2 a 4. `lipsum[1-1]` solo imprimirá lorem ipsum párrafo 1.




\section{Materiales y Métodos}
\lipsum[5] %idem comentario anterior (linea 16)
\section{Data Sets}










\end{document}
